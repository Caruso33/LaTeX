% https://www.uni-ulm.de/fileadmin/website_uni_ulm/mawi.inst.160/pdf_dokumente/LaTeX.pdf

%
% preamble
%
\documentclass[12pt,titlepage]{article}

\usepackage[ngerman]{babel}
\usepackage[utf8]{inputenc}
\usepackage{color}
\usepackage[a4paper,lmargin={4cm},rmargin={2cm},tmargin={2.5cm},bmargin = {2.5cm}]{geometry}
\usepackage{amssymb}
\usepackage{amsthm}
\usepackage{graphicx}
\usepackage{hyperref}
\usepackage{booktabs}

\usepackage[backend=bibtex]{biblatex}
\addbibresource{literatur.bib}

\linespread{1.25}
\def\theequation{\thesection.\arabic{equation}}

%
% doc start
%
\begin{document}


\pagenumbering{Roman}

\begin{titlepage}
    \title{Die Leiden des jungen Werthers}
    \date{1774}
    \author{Johann Wolfgang von Goethe}
    \maketitle
\end{titlepage}

\tableofcontents
\newpage

\pagenumbering{arabic}

\section{Section}
\subsection{Subsection}
\subsubsection{Subsubsection}

\begin{equation}
    E = mc^2
\end{equation}

$E = mc^2$ ist die zentrale Gleichung der Relativitätstheorie.

In expected utility theory, the utilities of outcomes are weighted by their probabilities\footfullcite[vgl.][]{kahneman}

\newpage
\begin{figure}
	\centering\includegraphics[width=0.25\textwidth]{eistüte.jpg}
	\caption[Eistüte]{Eistüte,\\ Quelle: eigene Darstellung}
\end{figure}


\begin{table}
\centering
\begin{tabular}{lcr} 
\midrule 
Land & Hauptstadt & Fläche
\\ \toprule 
Deutschland & Berlin & 357.111 $\mbox{km}^2$\\
Frankreich & Paris & 674.843 $\mbox{km}^2$\\ 
\midrule 
\end{tabular}

\caption{Staatsfläche,\\ Quelle: Wikipedia}
\end{table}

%
% appendix
%
\newpage
\addcontentsline{toc}{section}{Literatur}
\printbibliography

\newpage
\addcontentsline{toc}{section}{Abbildungsverzeichnis}
\listoffigures

\newpage
\addcontentsline{toc}{section}{Tabellenverzeichnis}
\listoftables

\end{document}